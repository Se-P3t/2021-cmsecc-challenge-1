\documentclass[10pt,compress]{beamer}
\RequirePackage{geometry}
\geometry{paperheight=9cm,paperwidth=16cm}

\usetheme{metropolis} % https://github.com/matze/mtheme
\setbeamertemplate{bibliography item}[text]
\setbeamercovered{transparent}

\usepackage{metropolis_framesubtitle}
\definecolor{black}{HTML}{000000}
\setbeamercolor{frametitle}{bg=black}
\setbeamercolor{framesubtitle}{fg=black, bg=normal text.bg}

\usepackage[UTF8]{ctex}
\usepackage{xspace}
%\usepackage{comment}
\usepackage{amsmath,amsfonts,amssymb,amsthm}
%\usepackage{graphicx}
%\usepackage{listings}



\newcommand{\Z}{\ensuremath{\mathbb{Z}}\xspace}
\newcommand{\LSB}[2]{\ensuremath{\textnormal{LSB}_{#1}\left(#2\right)}\xspace}
\newcommand{\MSB}[2]{\ensuremath{\textnormal{MSB}_{#1}\left(#2\right)}\xspace}
\newcommand{\bitlength}[1]{\ensuremath{\textnormal{bitlength}\left(#1\right)}\xspace}
\newcommand{\mat}[1]{\mathbf{#1}\xspace}
\newcommand{\leftkernel}[1]{\ensuremath{\textnormal{Ker}_{\textnormal{left}}\left(#1\right)}\xspace}
\newcommand{\norm}[1]{\left\lVert#1\right\rVert}
\newcommand{\norminf}[1]{\left\lVert#1\right\rVert_\infty}

\renewcommand{\vec}[1]{\overrightarrow{\mathbf{#1}}\xspace}

%\newcommand{\code}{\lstinline}


%\lstdefinelanguage{Sage}[]{Python}{
%  morekeywords={True,False,sage},
%  sensitive=true
%}
%
%\lstset{
%  frame=none,
%  showtabs=False,
%  showspaces=False,
%  showstringspaces=False,
%  language = Sage,
%  basicstyle={\tt}
%}


\usepackage[
  backend=bibtex,
  style=alphabetic,
  maxnames=4,
  citestyle=alphabetic
]{biblatex}

\addbibresource{refs.bib}


\AtBeginSection[] {
  \begin{frame}
    \frametitle{目录}
    \tableofcontents[sectionstyle=show/shaded,subsectionstyle=show/shaded]
  \end{frame}
}


\title{第六届(2021年)全国高校密码数学挑战赛}
\subtitle{赛题一:环上序列的截位还原}
\author{方佳乐 @$\href{https://github.com/Se-P3t}{Se\_P3t}$\\
 指导老师: 钟华}
\institute{杭州电子科技大学\, 通信工程学院\\
 信息安全协会\, \texttt{@Vidar-Team}}
\date{2021年8月23日}

\begin{document}

\begin{frame}
  \maketitle
\end{frame}

\setcounter{tocdepth}{1}
\begin{frame}
  \frametitle{目录}
  \tableofcontents
\end{frame}



\section{赛题介绍}

\begin{frame}{基本概念}
  \begin{block}{环上序列的截位还原问题}
    对于一个在整数剩余类环 $\Z / (m)$ 上的 $n$ 阶本原多项式 $f(x) = x^n - c_{n-1} x^{n-1} - \cdots - c_1 x - c_0$, 其对应的线性递归生成器定义为
    $$
    a_{i+n} = c_{n-1} a_{i+n-1} + c_{n-2} a_{i+n-2} + \cdots + c_0 a_i \bmod{m}
    $$
    其中 $a_i$ 为生成器的内部状态. 令 $a_i = 2^k y_i + z_i$ 且 $z_i = \LSB{k}{a_i} = a_i \bmod 2^k$, 那么该生成器的截位高 $\bitlength{m} - k$ 比特的输出为 $y_i$. 环上序列的截位还原问题定义为: \\
    给出生成器的一段长为 $d$ 的连续的截位输出 $y_0, y_1, \ldots, y_{d-1}$, 求出初态 $a_0, a_1, \ldots, a_{n-1}$.
  \end{block}
\end{frame}

\begin{frame}{赛题描述}
  \begin{description}
    \item[第一类挑战] 在 \underline{模数 $m$, 本原多项式 $f(x)$} 均已知的条件下, 求解环上序列的截位还原问题.
    \item[第二类挑战] 在 \underline{级数 $n$, 模数 $m$} 均已知的条件下, 求解环上序列的截位还原问题.
    \item[第三类挑战] 在 \underline{级数 $n$ 和 $m$ 的比特数} 均已知的条件下, 求解环上序列的截位还原问题.
  \end{description}
\end{frame}



\section{赛题分析}

\begin{frame}{赛题分析: 第一类挑战}
  Sun 等人在 \cite[Section 3.5]{Sun2020} 中将生成器内部状态之间的关系转换成一组 \underline{线性同余方程组}, 并考虑使用 Frieze 等人 \cite{Frieze1988} 的理论求解初态.
  \medbreak
  \pause
  令
  $$
  \mat{Q} = \begin{pmatrix}
    0 &\cdots &0 &c_0 \\
    1 &\cdots &0 &c_1 \\
    \vdots &\ddots &\vdots &\vdots \\
    0 &\cdots &0 &c_{n-1}
  \end{pmatrix},
  $$
  \smallbreak
  \pause
  则有 $\left(a_{i+j}, a_{i+j+1}, \ldots, a_{i+j+n-1}\right) = \left(a_i, a_{i+1}, \ldots, a_{i+n-1}\right) \mat{Q}^j \bmod{m}$
\end{frame}

\begin{frame}{赛题分析: 第一类挑战}
  设 $\mat{Q}^j$ 的第一列为 $\left(q_{j,0}, q_{j,1}, \ldots, q_{j,n-1}\right)^T$, 有任一状态 $a_j$ 关于初态的线性同余关系式:
  $$
  a_j = \sum_{l=0}^{n-1} q_{j,l} a_l \bmod{m}.
  $$
  \pause
  现给出截位高 $\bitlength{m} - k$ 比特输出 $y_i$, 代入得:
  $$
  c_j := 2^k \left(y_j - \sum_{l=0}^{n-1} q_{j,l} y_l\right) \bmod{m} \equiv \sum_{l=0}^{n-1} q_{j,l} z_l - z_j \pmod{m},
  $$
  其中 $0 \leq z_i < 2^k$.
\end{frame}

\begin{frame}{赛题分析: 第一类挑战}
  我们基于该线性同余方程组, 考虑了以下优化:
  \begin{itemize}
    \pause
    \item 利用嵌入技术转化成最短向量问题
    \pause
    \item 使用更高级的格基约化算法, 如 BKZ 算法
    \pause
    \item 调用筛法 \href{https://github.com/fplll/g6k}{G6K} 与 \href{https://github.com/WvanWoerden/G6K-GPU-Tensor}{G6K-GPU}
  \end{itemize}
\end{frame}

\begin{frame}{赛题分析: 第二,三类挑战}
  Sun 等人的解法不适用于第二类挑战, 在其研究基础上, 我们发现了关于初态的 \underline{新的线性关系}, 并由此提出一种基于零空间的解法能够直接求出初态.
  \bigbreak
  \pause
  对于第三类挑战, 由于格基维数过大, 我们只能求出前三级挑战.
\end{frame}



\section{解题思路}

\begin{frame}{解题思路: 第一类挑战}
  首先, 我们应用 Kannan 的嵌入技术 \cite{Kannan1987} 将其转换成 SVP 问题:
  \medbreak
  \pause
  $$
  \begin{aligned}
    \begin{pmatrix}
      -1 \\ z_0 \\ \vdots \\ z_{n-1} \\ k_n \\ \vdots \\ k_{d-1}
    \end{pmatrix}^T
    &\cdot
    \begin{pmatrix}
      \quad 1 &\quad &\quad &\quad &c_n &\cdots &c_{d-1} \\
      &1 & & &q_{n,0} &\cdots &q_{d-1,0} \\
      & &\ddots & &\vdots &\cdots &\vdots \\
      & & &1 &q_{n,n-1} &\cdots &q_{d-1,n-1} \\
      & & & &m & & \\
      & & & & &\ddots & \\
      & & & & & &m \\
    \end{pmatrix} \\
    &=
    \begin{pmatrix}
      -1 &z_0 &\ldots &z_{n-1} &z_n &\quad\ldots &\quad z_{d-1}
    \end{pmatrix}
  \end{aligned}
  $$
\end{frame}

\begin{frame}{解题思路: 第一类挑战}
  注意到目标向量中 $z_j$ 均大于等于零, 因此我们可以先进行换元以减少目标向量的长度一比特.\\
  \pause
  令 $\hat{z_j} = z_j - 2^{k-1}$, 则有 $\hat{z_j} \in \left[-2^{k-1}, 2^{k-1}\right)$.
  \medbreak
  \pause
  $$
  \begin{aligned}
    \begin{pmatrix}
      -1 \\ z_0 \\ \vdots \\ z_{n-1} \\ k_n \\ \vdots \\ k_{d-1}
    \end{pmatrix}^T
    &\cdot
    \begin{pmatrix}
      \quad 1 &2^{k-1} &\cdots &2^{k-1} &c_n+2^{k-1} &\cdots &c_{d-1}+2^{k-1} \\
      &1 & & &q_{n,0} &\cdots &q_{d-1,0} \\
      & &\ddots & &\vdots &\cdots &\vdots \\
      & & &1 &q_{n,n-1} &\cdots &q_{d-1,n-1} \\
      & & & &m & & \\
      & & & & &\ddots & \\
      & & & & & &m \\
    \end{pmatrix} \\
    &=
    \begin{pmatrix}
      -1 &\;\hat{z_0} &\quad\ldots &\space\hat{z_{n-1}} &\quad\hat{z_n} &\qquad\ldots &\qquad \hat{z_{d-1}}
    \end{pmatrix}
  \end{aligned}
  $$
\end{frame}

\begin{frame}{解题思路: 第一类挑战}
  最后, 根据目标向量预期大小, 我们可以取放缩因子为 $(m, \alpha, \ldots, \alpha)$, 其中 $\alpha = \lfloor m/2^{k-1}\rfloor$.
  \medbreak
  \pause
  $$
  \text{基矩阵}\, \mat{B} =
  \begin{pmatrix}
    m &m &\cdots &m &\alpha\left(c_n+2^{k-1}\right) &\cdots &\alpha\left(c_{d-1}+2^{k-1}\right) \\
    &1 & & &\alpha q_{n,0} &\cdots &\alpha q_{d-1,0} \\
    & &\ddots & &\vdots &\cdots &\vdots \\
    & & &1 &\alpha q_{n,n-1} &\cdots &\alpha q_{d-1,n-1} \\
    & & & &\alpha m & & \\
    & & & & &\ddots & \\
    & & & & & &\alpha m \\
  \end{pmatrix}
  $$
\end{frame}

\begin{frame}{解题思路: 第二,三类挑战}
  对于第二,三类挑战, 本原多项式未被给出, 因此我们无法利用之前的线性方程组. \\
  \bigbreak
  \pause
  Sun 等人考虑通过已知的截位输出 $y_i$ 获得关于内部状态 $a_i$ 的新的线性关系 \cite[Section 3.1]{Sun2020}:
  $$
  \sum_{i=0}^{r-1} \eta_i a_{j+i} = \sum_{i=0}^{r-1} \eta_i y_{j+i} = 0 \quad\left(j=0,1,\ldots,t-1\right)
  $$
\end{frame}

\begin{frame}{解题思路: 第二,三类挑战}
  对于 $d$ 组符合上述条件的向量 $\vec{\eta} = {\eta_0, \eta_1, \ldots, \eta_{r-1}}$, 我们构造如下 $(r+t-1)\times td$ 的矩阵:
  $$
  \mat{M} =
  \begin{pmatrix}
    \eta_0^{(0)} &\cdots &\eta_0^{(d-1)} & & & & \\
    \vdots &\cdots &\vdots &\ddots & & & \\
    \eta_{r-1}^{(0)} &\cdots &\eta_{r-1}^{(d-1)} &\ddots &\eta_0^{(0)} &\cdots &\eta_0^{(d-1)} \\
     & & &\ddots &\vdots &\cdots &\vdots \\
     & & & &\eta_{r-1}^{(0)} &\cdots &\eta_{r-1}^{(d-1)}
  \end{pmatrix}
  $$
  \pause
  显然, 向量 $\vec{y} = {y_0, y_1, \ldots, y_{r+t-2}}$ 与向量 $\vec{z} = {z_0, z_1, \ldots, z_{r+t-2}}$ 满足
  $$
  \vec{y} \mat{M} = \vec{z} \mat{M} = \vec{0}
  $$
  即, $\vec{y}, \vec{z} \in \leftkernel{\mat{M}}$.
\end{frame}

\begin{frame}{解题思路: 第二,三类挑战}
  现在我们考虑由矩阵 $\mat{M}$ 的左核 $\leftkernel{\mat{M}}$ 生成的格 $L$. 一般的, 当 $d>(r+t-1)/t$ 时, $\mat{M}$ 的零空间维数恰为 2, 即格 $L$ 的一组基为 $\{\vec{y}, \vec{z}\}$. \\
  \pause
  设 $\mat{B}$ 为格 $L$ 的 LLL 规约基, $\text{ybit} = \max{\bitlength{y_i}}$, $\text{zbit} = \max{\bitlength{z_i}}$. 我们分以下四种情况讨论:
  \begin{itemize}
    \item zbit < ybit
    \item ybit $\leq$ zbit < ybit + 2
    \item zbit > ybit + 2
    \item zbit = ybit + 2
  \end{itemize}
\end{frame}

\begin{frame}{解题思路: 第二,三类挑战}{zbit < ybit}
  格维数为 2, 且 $\norm{\vec{z}} \ll \norm{\vec{y}}$, 那么经过 LLL 规约后, 有
  $$
  \mat{B}[0] = \vec{z}
  $$
\end{frame}

\begin{frame}{解题思路: 第二,三类挑战}{ybit $\leq$ zbit < ybit + 2}
  对输出进行代换: 取 $\vec{y'} = 2\vec{y}+1$, 则有 $\vec{z'} = \left(z_0-2^{\text{zbit}-1}, \ldots, z_{r+t-1}-2^{\text{zbit}-1}\right)$, 同上求出多组 $\vec{\eta'}$ 构造矩阵 $\mat{M'}$ 并设 $\leftkernel{\mat{M'}}$ 生成的格 $L'$ 的 LLL 规约基为 $\mat{B'}$. \\
  \bigbreak
  \pause
  相应地, 我们有 $\bitlength{\norminf{\vec{y'}}} = \text{ybit} + 1$, $\bitlength{\norminf{\vec{z'}}} = \text{zbit} - 1$, \\
  \medbreak
  即 $\norm{\vec{z'}} \ll \norm{\vec{y'}}$, 亦即
  $$
  \mat{B'}[0] = \vec{z'}
  $$
\end{frame}

\begin{frame}{解题思路: 第二,三类挑战}{zbit > ybit + 2}
  此时 $\norm{\vec{y}} < \norm{\vec{z}}$, $\norm{\vec{y'}} < \norm{\vec{z'}}$ \\
  \pause
  存在一幺模矩阵 $\mat{U}$ 使得
  $$
  \mat{B} = \mat{U} \cdot
  \begin{pmatrix}
    \vec{y} \\
    \vec{z}
  \end{pmatrix}
  $$
  不失一般性, 我们有
  $$
  \mat{U} =
  \begin{pmatrix}
    1 & \\
    k &\pm 1
  \end{pmatrix},
  $$
  其中 $k \in \Z$. \\
  \pause
  综上, 我们有 $\{\vec{y}, \vec{z}+k_1\vec{y}, \vec{y'}, \vec{z'}+k_2\vec{y'}\}$
\end{frame}

\begin{frame}{解题思路: 第二,三类挑战}{zbit > ybit + 2}
  我们有 $\{\vec{y}, \vec{z}+k_1\vec{y}, \vec{y'}, \vec{z'}+k_2\vec{y'}\}$ \\
  \bigbreak
  \bigbreak
  \bigbreak
  \pause
  给最后一个向量加上 $\left(2^{\text{zbit}-1}, \ldots, 2^{\text{zbit}-1}\right)$ 得到 $\vec{z} + k_2\vec{y'}$. \\
  \pause
  注意到
  $$
  \vec{z} + k_2\vec{y'} = (\vec{z}+k_1\vec{y}) - k_1(\vec{y}) + k2(\vec{y'})
  $$
  所以我们可以求出 $k_1, k_2$, 最终获得 $\vec{z}$.
\end{frame}

\begin{frame}{解题思路: 第二,三类挑战}{zbit = ybit + 2}
  挑战中并未出现该情形, 这里不做进一步讨论. \\
  \pause
  \bigbreak
  \bigbreak
  \begin{itemize}
    \item 试试运气
    \item 尝试其他代换
  \end{itemize}
\end{frame}

\begin{frame}{解题思路: 第二,三类挑战}
  在求出初态后 \\
  \begin{description}
    \pause
    \item[求模数 $m$] 计算 GCD 直到比特数一致
    \pause 
    \item[求系数 $c_i$] 获得 $n$ 组关于矩阵 $\mat{Q}$ 的等式, 在 $\Z / (m)$ 上求解
  \end{description}
\end{frame}



\section{方案亮点}

\begin{frame}{方案亮点}
  \begin{itemize}
    \pause
    \item 第一类挑战转换成 SVP 问题, 简化了过程
    \pause
    \item 第二,三类挑战直接求出初态, 只需要一两次BKZ, 缩短了时间
    \pause
    \item 使用了 GPU-G6K, 与时俱进, 大幅缩短了时间
    \pause
    \item 代码完善, 提供了测试与验证代码, 方便进一步的研究
    \pause
    \item 封装了 lattice 与 meg 类, 简化了重复代码, 调用方便
  \end{itemize}
\end{frame}



\section{研究成果}

\begin{frame}{研究成果}{第一类挑战}
  第一类挑战我们主要使用 FPyLLL \cite{fpylll} 中 BKZ 算法实现以及 G6K \cite{Albrecht2019a} 多核 CPU 筛法;
  \pause
  其中, 对于第七级挑战, 我们使用 GPU 筛法 \cite{Ducas2021}, 并在 Google Colab 上于两小时内求出结果. \\
  \pause
  具体解题时间如下:
  \begin{table}
    \begin{tabular}{l | c | c | c }
      级数 &求解算法 &格基维数 &耗费时间 \\
      \hline \hline
      1 &HKZ &19 &<1s \\
      2 &HKZ &20 &<1s \\
      3 &BKZ-20 &31 &$\approx$1s \\
      4 &BKZ-40 &52 &$\approx$1s \\
      5 &BKZ-40 &74 &5.39s \\
      6 &BKZ-30 + Sieve &121 &2m13s \\
      7 &BKZ-20 + GPU Sieve &151 &99m4s
    \end{tabular}
    \caption{第一类挑战解题时间}
  \end{table}
\end{frame}

\begin{frame}{研究成果}{第二类挑战}
  在第二类挑战中, 我们提出了基于零空间的解法, 能够在数秒内求解出所有九级挑战. \\
  \pause
  \begin{table}
    \begin{tabular}{l | c | c | c | c | c | c }
      级数 &阶数 &zbits &r &t &BKZ block size &耗费时间 \\
      \hline \hline
      1 &2 &17 &30 &8 &20 &2.35s \\
      2 &2 &23 &60 &15 &20 &2.9s \\
      3 &3 &21 &68 &17 &20 &3.2s \\
      4 &4 &21 &95 &25 &30 &7.8s \\
      5 &5 &18 &85 &23 &30 &6.7s \\
      6 &8 &11 &90 &20 &20 &4.4s \\
      7 &10 &11 &110 &26 &20 &9.3s \\
      8 &12 &8 &110 &28 &20 &4.65s \\
      9 &14 &8 &128 &32 &32 &10.0s
    \end{tabular}
    \caption{第二类挑战解题时间}
  \end{table}
\end{frame}

\begin{frame}{研究成果}{第三类挑战}
  对于三类挑战, 由于维数过大和时间限制, 我们只能解出前三级挑战: \\
  \pause
  \begin{table}
    \begin{tabular}{l | c | c | c | c | c | c | c }
      级数 &阶数 &mbits &zbits &r &t &求解算法 &耗费时间 \\
      \hline \hline
      1 &16 &31 &5 &140 &30 &BKZ-20 &30.8s \\
      2 &16 &31 &10 &190 &40 &BKZ-30 &1m14.8s \\
      3 &16 &31 &14 &265 &70 &BKZ-30+Sieve &5m24.0s
    \end{tabular}
    \caption{第三类挑战解题时间}
  \end{table}
\end{frame}



\begin{frame}{致谢}
  \begin{center}
    \Huge 谢\ 谢!
  \end{center}
\end{frame}



\section{参考文献}

\begin{frame}[allowframebreaks]{参考文献}
  \nocite{*}
  \printbibliography[heading=none]
\end{frame}


\end{document}